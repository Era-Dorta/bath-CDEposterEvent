%%%%%%%%%%%%%%%%%%%%%%%%%%%%%%%%%%%%%%%%%
% baposter Landscape Poster
% LaTeX Template
% Version 1.0 (11/06/13)
%
% baposter Class Created by:
% Brian Amberg (baposter@brian-amberg.de)
%
% This template has been downloaded from:
% http://www.LaTeXTemplates.com
%
% License:
% CC BY-NC-SA 3.0 (http://creativecommons.org/licenses/by-nc-sa/3.0/)
%
%%%%%%%%%%%%%%%%%%%%%%%%%%%%%%%%%%%%%%%%%

%----------------------------------------------------------------------------------------
%	PACKAGES AND OTHER DOCUMENT CONFIGURATIONS
%----------------------------------------------------------------------------------------

\documentclass[landscape,a0paper,fontscale=0.285]{baposter} % Adjust the font scale/size here

\usepackage{graphicx} % Required for including images
\graphicspath{{figures/}} % Directory in which figures are stored

\usepackage{amsmath} % For typesetting math
\usepackage{amssymb} % Adds new symbols to be used in math mode

\usepackage{booktabs} % Top and bottom rules for tables
\usepackage{enumitem} % Used to reduce itemize/enumerate spacing
\usepackage{palatino} % Use the Palatino font
\usepackage[font=small,labelfont=bf]{caption} % Required for specifying captions to tables and figures

\usepackage{multicol} % Required for multiple columns
\setlength{\columnsep}{1.5em} % Slightly increase the space between columns
\setlength{\columnseprule}{0mm} % No horizontal rule between columns

\usepackage{tikz} % Required for flow chart
\usetikzlibrary{shapes,arrows} % Tikz libraries required for the flow chart in the template

\newcommand{\compresslist}{ % Define a command to reduce spacing within itemize/enumerate environments, this is used right after \begin{itemize} or \begin{enumerate}
\setlength{\itemsep}{1pt}
\setlength{\parskip}{0pt}
\setlength{\parsep}{0pt}
}

\definecolor{lightblue}{rgb}{0.145,0.6666,1} % Defines the color used for content box headers

\begin{document}

\begin{poster}
{
columns=3,
headerborder=closed, % Adds a border around the header of content boxes
colspacing=1em, % Column spacing
bgColorOne=white, % Background color for the gradient on the left side of the poster
bgColorTwo=white, % Background color for the gradient on the right side of the poster
borderColor=lightblue, % Border color
headerColorOne=black, % Background color for the header in the content boxes (left side)
headerColorTwo=lightblue, % Background color for the header in the content boxes (right side)
headerFontColor=white, % Text color for the header text in the content boxes
boxColorOne=white, % Background color of the content boxes
textborder=roundedleft, % Format of the border around content boxes, can be: none, bars, coils, triangles, rectangle, rounded, roundedsmall, roundedright or faded
eyecatcher=true, % Set to false for ignoring the left logo in the title and move the title left
headerheight=0.1\textheight, % Height of the header
headershape=roundedright, % Specify the rounded corner in the content box headers, can be: rectangle, small-rounded, roundedright, roundedleft or rounded
headerfont=\Large\bf\textsc, % Large, bold and sans serif font in the headers of content boxes
%textfont={\setlength{\parindent}{1.5em}}, % Uncomment for paragraph indentation
linewidth=2pt % Width of the border lines around content boxes
}
%----------------------------------------------------------------------------------------
%	TITLE SECTION 
%----------------------------------------------------------------------------------------
%
{\includegraphics[height=6em]{cde_tag_black.png}} % First university/lab logo on the left
{\bf\textsc{Doctor of Engineering (EngD) in Digital Entertainment, 1st year}\vspace{0.5em}} % Poster title
{\textsc{\{ Garoe Dorta Perez, Ieva Kazlauskaite and Richard Shaw \} \hspace{12pt} University of Bath}} % Author names and institution
{\includegraphics[height=5em]{Bathlogo.jpg}} % Second university/lab logo on the right

%----------------------------------------------------------------------------------------
%	OBJECTIVES
%----------------------------------------------------------------------------------------

\headerbox{Inverse kinematics}{name=objectives,column=0,row=0}{

\begin{center}
\includegraphics[width=0.8\linewidth]{chain3Dexamplev2}
\captionof{figure}{Figure caption}
\end{center}

Inverse kinematics uses a known end-effector position $\mathbf{E}$ to calculate the angles between the parts   $\boldsymbol\theta$ which ensure that the object reaches this desired target position, such that

\begin{align*}
\mathbf{E} &= f(\boldsymbol\theta) \rightarrow \boldsymbol{\theta} = f^{-1}(\mathbf{E}) \\  \partial \mathbf{E} &\approx J(\boldsymbol\theta) \partial \boldsymbol\theta \rightarrow \partial \boldsymbol{\theta} \approx J^{-1}(\partial\mathbf{E}),
\end{align*}

where $f$ is the forward kinematics solver and $J$ is the Jacobian matrix.

\vspace{0.3em} % When there are two boxes, some whitespace may need to be added if the one on the right has more content
}

%----------------------------------------------------------------------------------------
%	INTRODUCTION
%----------------------------------------------------------------------------------------

\headerbox{Fonts Regression}{name=introduction,column=1,row=0,bottomaligned=objectives}{


}

%----------------------------------------------------------------------------------------
%	RESULTS 1
%----------------------------------------------------------------------------------------

\headerbox{SIFT Features}{name=results,column=2,span=1,row=0}{

\vspace{1em}
\begin{center}
\includegraphics[width=0.8\linewidth]{placeholder}
\captionof{figure}{Figure caption}
\end{center}

Aliquam auctor, metus id ultrices porta, risus enim cursus sapien, quis iaculis sapien tortor sed odio. Mauris ante orci, euismod vitae tincidunt eu, porta ut neque. Aenean sapien est, viverra vel lacinia nec, venenatis eu nulla. Maecenas ut nunc nibh, et tempus libero. Aenean vitae risus ante. Pellentesque condimentum dui. Etiam sagittis purus non tellus tempor volutpat. Donec et dui non massa tristique adipiscing.

}

%----------------------------------------------------------------------------------------
%	REFERENCES
%----------------------------------------------------------------------------------------

\headerbox{References}{name=references,column=0,above=bottom}{

\renewcommand{\section}[2]{\vskip 0.05em} % Get rid of the default "References" section title
\nocite{*} % Insert publications even if they are not cited in the poster
\small{ % Reduce the font size in this block
\bibliographystyle{unsrt}
\bibliography{cde_poster} % Use sample.bib as the bibliography file
\vspace{1em}
}}

%----------------------------------------------------------------------------------------
%	FUTURE RESEARCH
%----------------------------------------------------------------------------------------

\headerbox{Future Research}{name=futureresearch,column=1,span=1,aligned=references,above=bottom}{ % This block is as tall as the references block

Integer sed lectus vel mauris euismod suscipit. Praesent a est a est ultricies pellentesque. Donec tincidunt, nunc in feugiat varius, lectus lectus auctor lorem, egestas molestie risus erat ut nibh.
\vspace{1em}
}

%----------------------------------------------------------------------------------------
%	CONTACT INFORMATION
%----------------------------------------------------------------------------------------

\headerbox{Contact Information}{name=contact,column=2,aligned=references,above=bottom}{ % This block is as tall as the references block

\begin{description}\compresslist
\item[Web] http://www.digital-entertainment.org
\item[Garoe Dorta Perez] g.dorta.perez@bath.ac.uk 
\item[Ieva Kazlauskaite] ik359@bath.ac.uk 
\item[Richard Shaw] richard.o.shaw@gmail.com
\end{description}
}



%----------------------------------------------------------------------------------------
%	CONCLUSION
%----------------------------------------------------------------------------------------

\headerbox{Visual Understanding 2}{name=conclusion,column=2,span=1,row=0,below=results,above=references}{

\begin{itemize}\compresslist
\item Pellentesque eget orci eros. Fusce ultricies, tellus et pellentesque fringilla, ante massa luctus libero, quis tristique purus urna nec nibh. Phasellus fermentum rutrum elementum. Nam quis justo lectus.
\item Vestibulum sem ante, hendrerit a gravida ac, blandit quis magna.
\item Donec sem metus, facilisis at condimentum eget, vehicula ut massa. Morbi consequat, diam sed convallis tincidunt, arcu nunc.
\item Nunc at convallis urna. isus ante. Pellentesque condimentum dui. Etiam sagittis purus non tellus tempor volutpat. Donec et dui non massa tristique adipiscing.
\end{itemize}

}

%----------------------------------------------------------------------------------------
%	MATERIALS AND METHODS
%----------------------------------------------------------------------------------------

\headerbox{Shape interpolation}{name=method,column=0,below=objectives,bottomaligned=conclusion}{ % This block's bottom aligns with the bottom of the conclusion block

The following materials were required to complete the research:

\begin{itemize}\compresslist
\item Curabitur pellentesque dignissim
\item Eu facilisis est tempus quis
\item Duis porta consequat lorem
\item Eu facilisis est tempus quis
\end{itemize}

The following equations were used for statistical analysis:

\begin{equation}
\cos^3 \theta =\frac{1}{4}\cos\theta+\frac{3}{4}\cos 3\theta
\label{eq:refname}
\end{equation}\

\begin{equation}
E = mc^{2}
\label{eqn:Einstein}
\end{equation}


}

%----------------------------------------------------------------------------------------
%	RESULTS 2
%----------------------------------------------------------------------------------------

\headerbox{3D Reconstruction}{name=results2,column=1,below=objectives,bottomaligned=conclusion}{ % This block's bottom aligns with the bottom of the conclusion block

Donec faucibus purus at tortor egestas eu fermentum dolor facilisis. Maecenas tempor dui eu neque fringilla rutrum. Mauris \emph{lobortis} nisl accumsan.

\begin{center}
\begin{tabular}{l l l}
\toprule
\textbf{Treatments} & \textbf{Response 1} & \textbf{Response 2}\\
\midrule
Treatment 1 & 0.0003262 & 0.562 \\
Treatment 2 & 0.0015681 & 0.910 \\
Treatment 3 & 0.0009271 & 0.296 \\
\bottomrule
\end{tabular}
\captionof{table}{Table caption}
\end{center}

Nulla ut porttitor enim. Suspendisse venenatis dui eget eros gravida tempor. Mauris feugiat elit et augue placerat ultrices. Morbi accumsan enim nec tortor consectetur non commodo.

}

%----------------------------------------------------------------------------------------

\end{poster}

\end{document}